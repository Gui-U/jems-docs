\def \tagQualiNom {Peredhel}
\def \tagQualiPrenom {Elrond}
\newcommand{\tagEtudeNum}{01001}
\newcommand{\tagDocNameShort}{CC}
\newcommand{\tagDocName}{Convention client \tagEtudeNum}
\newcommand{\tagDocDate}{10 janvier 1970}
%\newcommand{\tagEtudeGarantie}{3 mois}%{3mois} si étude technique et{15 jours} si rapport, étude de marché ou traduction

%\newcommand{\tagPrezNom}{Le Blanc}
%\newcommand{\tagPrezPrenom}{Gandalf}
%\newcommand{\tagPrezSexeCivilite}{M.}%{M.} or {Mme.}
%\newcommand{\tagPrezSexeE}{}%{e} or {}
%\newcommand{\tagPrezSexePossessif}{son}
\newcommand{\tagClientCompagny}{Consortium des diamants}
%\newcommand{\tagClientNom}{Nonyme}
%\newcommand{\tagClientPrenom}{A.}
%\newcommand{\tagClientFonction}{Gentil donateur}
%\newcommand{\tagClientAdresseRue}{Impasse des émeraudes}
%\newcommand{\tagClientAdresseCodePostal}{75042}
%\newcommand{\tagClientAdresseVille}{El Dorado}
%\newcommand{\tagDocCity}{\tagJeAdresseVille}
%\def \typeDoc {cc}
%% \begin{filecontents*}{real.csv} %https://tex.stackexchange.com/questions/146716/importing-csv-file-into-latex-as-a-table
%%   name,givenname
%%   A,Arthur
%%   B,Bastien
%%   C,Corentin
%% \end{filecontents*}
\newcommand{\tagCommercialNom}{Skywalker}
\newcommand{\tagCommercialPrenom}{Anakin}
\newcommand{\tagDocVersion}{1.0}

